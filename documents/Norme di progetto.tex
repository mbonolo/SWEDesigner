%Document-Author: Cognome Nome + Cognome Nome + ...
%Document-Date: AAAA-MM-GG
%Document-Description: Descrizione documento


\documentclass[a4paper]{report}
\usepackage[english, italian]{babel}
\usepackage[T1]{fontenc}
\usepackage[utf8]{inputenc}
\usepackage{url}
\usepackage{graphicx}
\graphicspath{{../Figures/}}
\usepackage[hidelinks]{hyperref}
\usepackage{booktabs}
\usepackage{tabularx}
\usepackage{pifont}
\usepackage[table]{xcolor}
\usepackage{float}
\newcommand{\mail}[1]{\href{mailto:#1}{\texttt{#1}}}	% per scrivere indirizzi e-mail

\newcolumntype{s}{>{\hsize=.21\hsize}X}
\newcolumntype{f}{>{\hsize=.37\hsize}X}
\newcolumntype{m}{>{\hsize=.42\hsize}X}

\newcommand{\mychapter}[2]{
	\setcounter{chapter}{#1}
	\setcounter{section}{0}
	\setcounter{subsection}{1}
	\chapter*{#2}
	\addcontentsline{toc}{chapter}{#2}
}

\renewcommand{\abstractname}{Tabella contenuti}

\begin{document}
	
	\begin{titlepage}
		% Defines a new command for the horizontal lines, change thickness here
		\newcommand{\HRule}{\rule{\linewidth}{0.5mm}} 
		\center  
		
		% HEADING SECTION
		\textsc{\LARGE KaleidosCode}\\[1.5cm] 
		\textsc{\Large SweDesigner}\\[0.5cm] 
		\textsc{\large Software per diagrammi UML}\\[0.5cm]
		
		
		% TITLE SECTION
		\HRule \\[0.4cm]
		{ \huge \bfseries Norme di progetto}\\[0.4cm] 
		\HRule \\[1.5cm]
		
		% AUTHOR SECTION
		\begin{minipage}{0.4\textwidth}
			\begin{flushleft} \large
				\emph{Redattori:}\\
				Bonato Enrico\\
				Bonolo Marco\\
				Pace Giulio\\
				Pezzuto Francesco\\
			\end{flushleft}
		\end{minipage}
		~
		\begin{minipage}{0.4\textwidth}
			\begin{flushright} \large
				\emph{Approvazione:} \\
				Sovilla Matteo\\
				\emph{Verifica:} \\
				Sanna Giovanni
			\end{flushright}
		\end{minipage}
		
		%immagine
		\begin{figure}[H]
			\centering
			\includegraphics[width=\textwidth]{Figures/KaleidosCodeLogo.png}
		\end{figure}
		
		%destinazione d'uso e lista di distribuzione	
		\begin{minipage}{0.4\textwidth}
			\begin{flushleft} \large
				\emph{Uso:}\\
				Interno \\
				
			\end{flushleft}
		\end{minipage}
		~
		\begin{minipage}{0.4\textwidth}
			\begin{flushright} \large
				\emph{Lista di distribuzione:} \\
				Prof. Vardanega Tullio\\
				Prof. Cardin Riccardo\\
				Zucchetti spa
			\end{flushright}
		\end{minipage}
		
		\begin{center}
			Versione 1.0.0
		\end{center}
		% Date, change the \today to a set date if you want to be precise
		{\large \today}\\[3cm] 
		% Fill the rest of the page with whitespace
		\vfill  
	\end{titlepage}
	
	
	\tableofcontents
	
	\mychapter{0}{Diario delle modifiche}
		\begin{table}[H]
			\begin{tabularx}{\textwidth}{s f m X}
				\noalign{\hrule height 1.5pt}
				\rowcolor{orange!85} Versione & Data & Autore & Descrizione \\
				\noalign{\hrule height 1.5pt}
				1.0.0 & 2017/03/01 & Bonolo Marco & Stesura capitolo Riunioni\\
				1.0.0 & 2017/03/01 & Pezzuto \hbox{Francesco} & Stesura
				capitoli Introduzione e Comunicazioni\\
				1.0.0 & 2017/03/01 & Pace Giulio & Creazione scheletro del documento e stesura parziale dei documenti\\
			\end{tabularx}
			\caption{Diario delle modifiche \label{tab:table_label}}
		\end{table}

	\mychapter{1}{Introduzione}
	\section{Scopo del documento}
		Questo documento definisce le norme che i membri del gruppo
		KaleidosCode adotteranno nello svolgimento del progetto
		``SWEDesigner''.\\
		Tutti i membri del gruppo sono tenuti a leggere il documento
		e a seguirne le norme descritte per uniformare il materiale
		prodotto, ridurre il numero di errori e migliorare l'efficienza.\\
		In particolare verranno definite norme riguardanti:
		\begin{itemize}
		\item Interazioni tra i membri del gruppo;
		\item Stesura di documenti e convenzioni;
		\item Modalità di lavoro durante le varie fasi del progetto;
		\item Ambiente di lavoro.
		\end{itemize}
	\section{Scopo del prodotto}
		Lo scopo del progetto è la realizzazione di un software di
		costruzione di diagrammi UML\ped{G} con la relativa generazione
		di codice Java\ped{G} e Javascript\ped{G} utilizzando tecnologie
		web.
	\section{Glossario}
		Al fine di evitare ogni ambiguità di linguaggio e massimizzare la
		comprensione dei documenti, i termini tecnici, di dominio, gli
		acronimi e le parole che necessitano di essere chiarite, sono
		riportate nel documento \textit{Glossario v1.0.0}.\\
		Ogni occorrenza di vocaboli presenti nel \textit{Glossario} è
		marcata da una ``G'' maiuscola in pedice.
	\section{Riferimenti}
		\subsection{Informativi}	
			\begin{itemize}
			\item\textbf{Specifiche UTF-8}\ped{G}:\\
			\url{http://www.unicode.org/versions/Unicode6.1.0/ch03.pdf};
			\item\textbf{ISO}\ped{G} \textbf{8601}:
			\url{it.wikipedia.org/wiki/ISO_8601};
			\item\textbf{Piano di Progetto}: \textit{Piano di Progetto
			v1.0.0};
			\item\textbf{Piano di Qualifica}: \textit{Piano di Qualifica
			v1.0.0}.
			\end{itemize}
	
	\mychapter{2}{Comunicazioni}
	\section{Comunicazioni esterne}
		Per le comunicazioni esterne è stata creata la casella di posta
		elettronica:
		\begin{center}
		\mail{kaleidos.codec6@gmail.com}
		\end{center}
		Tale indirizzo deve essere l'unico canale di comunicazione tra il
		gruppo di lavoro e l'esterno.
		Il \textit{Responsabile di Progetto} è l'unico ad accedere
		all'indirizzo ed è quindi l'unico a poter comunicare con il
		committente del progetto. È compito del \textit{Responsabile di
		Progetto} informare i membri del gruppo delle discussioni avvenute e,
		qualora fosse necessario, inoltrargli il messaggio attraverso
		una mailing list\ped{G}.
	\section{Comunicazioni interne}
		Per le comunicazioni interne viene utilizzato il sistema di
		comunicazione offerto in Asana.\\
		Tale sistema deve essere utilizzato dai membri del gruppo
		per comunicare tra loro. Tutte le conversazioni devono avere
		come destinatario l'indirizzo ``Kaleidos Code''.
		In questo modo, ogni componente è costantemente informato sullo
		scambio di informazioni interne.
		Qualora fosse necessario l'uso di e-mail, come ad esempio nel caso di
		un inoltro di messaggio da parte del \textit{Responsabile di
		Progetto}, è stata creata una mailing list:
		\begin{center}
		\mail{kaleidos.code@group.com}
		\end{center}
		Per facilitare le comunicazioni tra i membri del gruppo, viene
		utilizzato anche il sistema di messaggistica e videoconferenza
		Google Hangout.
		L'uso di quest'ultimo, nel caso in cui
		vengano prese decisioni	o emergano contenuti utili allo
		sviluppo del progetto, comporta l'obbligo di redigere un verbale
		da parte di un membro del gruppo,
		che pubblicherà attraverso il sistema di comunicazione di Asana una
		volta terminata la conversazione.
		La verbalizzazione ha l'obiettivo di tenere
		traccia di ogni argomento discusso, in
		quanto una comunicazione verbale non documentata non è
		accettabile per il corretto svolgimento del progetto.\\
		Per una comunicazione istantanea è utilizzato anche il sistema
		di messaggistica Telegram. Si richiede che la conversazione
		venga documentata come sopra descritto.
	\section{Composizione e-mail e conversazioni}
		In questo paragrafo viene descritta la struttura che deve avere
		un messaggio sia per una comunicazione esterna che per una
		conversazione interna attraverso il servizio offerto in Asana e
		mailing list\ped{G}.
		\subsection{Destinatario}
			\begin{itemize}
			\item\textbf{Interno - Asana}: l'unico indirizzo utilizzabile è
			il nome del gruppo: Kaleidos Code;
			\item\textbf{Interno - e-mail}: l'unico indirizzo utilizzabile è
			\mail{kaleidos.code@group.com};
			\item\textbf{Esterno}: può variare a seconda che ci si debba
			riferire  al Proponente, al Prof. Vardanega Tullio o al Prof.
			Cardin Riccardo.
			\end{itemize}
		\subsection{Mittente}
			\begin{itemize}
			\item\textbf{Interno - Asana}: è rappresentato automaticamente
			dallo username del creatore della conversazione;
			\item\textbf{Interno - e-mail}: l'indirizzo di chi scrive
			il messaggio;
			\item\textbf{Esterno}: l'unico indirizzo utilizzabile è
			\mail{kaleidos.codec6@gmail.com} e deve essere usato solamente dal
			\textit{Responsabile di Progetto}.
			\end{itemize}
		\subsection{Oggetto}
			L'oggetto deve essere chiaro ed esaustivo, possibilmente non
			confondibile con altri preesistenti.\\
			L'oggetto di una comunicazione,
			una volta avviata, non deve mai essere cambiato.\\
			Solamente per le e-mail, nel caso si debba
			comporre un messaggio di risposta vi è l'obbligo di aggiungere la
			particella ``Re:'' all'inizio dell'oggetto per poter distinguere i
			livelli di risposta; se si dovesse trattare di un inoltro, si deve
			usare invece la particella ``I:''.
		\subsection{Corpo}
			Il corpo di un messaggio deve contenere tutte le informazioni
			necessarie alla piena comprensione della comunicazione.\\
			Nel caso di e-mail, se alcune parti del messaggio hanno uno o più
			destinatari specifici,
			sarà necessario aggiungere il loro nome	prima del relativo paragrafo
			attraverso la segnatura	\textit{@destinatario};
			in Asana invece, si dovrà menzionare lo specifico destinatario
			attraverso la determinata funzionalità alla creazione del messaggio.
			\\Solamente per le e-mail, in caso di risposta od inoltro del
			messaggio, il contenuto aggiunto deve essere sempre messo in testa.
			Si consiglia di non cancellare il resto del messaggio,
			per consentire una visione completa della discussione.
		\subsection{Allegati}
			Qualora fosse necessario, è permesso l'uso di allegati. Possono
			essere usati ad esempio per inviare i verbali di una riunione dei
			membri del gruppo oppure di un incontro con il proponente o con
			il committente.
	
	\mychapter{3}{Riunioni}
	\section{Frequenza}
		Le riunioni interne del gruppo di lavoro avranno una cadenza settimanale.
		\dots
	\section{Convocazione riunione}
		\subsection{Interna}
			Il Responsabile di Progetto ha il compito di convocare le riunioni generali interne, ossia
			le riunioni a cui sono convocati tutti i membri del gruppo.
			Se un componente qualsiasi ritiene necessaria la convocazione di una riunione, deve
			inoltrare la richiesta al responsabile il quale decide se respingere o accettare
			tale richiesta.
			Il responsabile deve convocare l'assemblea con almeno quattro giorni di preavviso
			attraverso l'invio di una richiesta nella sezione Comunicazioni di Asana, il cui
			corpo è formato da:
			\begin{itemize}
				\item Oggetto: Convocazione riunione n. X (dove X indica il numero progressivo
				di riunioni effettuate)
				\item Corpo:
				\begin{itemize}
					\item Data e ora prevista
					\item Luogo previsto
					\item Ordine del giorno
				\end{itemize}
			\end{itemize}
			Ogni componente deve rispondere al messaggio nel minor tempo possibile confermando
			la presenza o in caso contrario motivando l'eventuale assenza. In caso di manca risposta
			il Responsabile di Progetto è obbligato a contattare direttamente il componente che non
			ha fornito una risposta. Una volta ricevute tutte le risposte, il Resposanbile di Progetto
			può decidere se confermare, rinviare o annullare la riunione, basandosi sul numero di
			presenze e/o assenze, per garantire il massimo numero di presenti. La conferma, il rinvio,
			l'annullamento devono essere effettuati tramite messaggio nella sezione Comunicazioni di cui
			sopra.
			\dots
		\subsection{Esterna}
			Il Responsabile di Progetto ha il compito di convocare le riunioni generali esterne, ossia
			le riunioni a cui sono convocati tutti i membri del gruppo e il proponente e/o committente.
			Se un componente qualsiasi ritiene necessaria la convocazione di una riunione, deve
			inoltrare la richiesta al responsabile il quale decide se respingere o accettare
			tale richiesta.
			Il responsabile deve prima di tutto concordare con il proponente e/o committente delle date e dei luoghi
			possibili in cui svolgere la riunione attraverso l'invio di un email contenente:
			\begin{itemize}
				\item Oggetto: Richiesta riunione
				\item Corpo
				\begin{itemize}
					\item Motivazione
					\item Eventuali date e/o luoghi possibili
				\end{itemize}
			\end{itemize}
			Dopo aver concordato delle date possibili, il responsabile deve inviare un messaggio a
			i membri del gruppo nella sezione Comunicazioni in Asana in cui sono specificate:
			\begin{itemize}
				\item Oggetto: Richiesta riunione col proponente
				\item Corpo
				\begin{itemize}
					\item Date e/o luoghi possibili
				\end{itemize}
			\end{itemize}
			Ogni membro del gruppo deve rispondere alla comunicazione confermando la presenza o in
			caso contrario motivando l'eventuale essenza. In caso di mancata risposta, il Responsabile
			di Progetto deve contattare direttamente il componente che non ha fornito una risposta.
			Una volta ricevute tutte le risposte, il responsabile può decidere se confermare, rinviare,
			annullare la riunione con il proponente basandosi sul numero di presenze e /o assenze.
			Il Responsabile di Progetto deve poi, in caso di:
			\begin{itemize}
				\item conferma: comunicare a tutti i membri e al proponente e/o committente la data e il luogo definitivi
				\item rinvio: comunicare a tutti i membri la decisione presa e ripetere il procedimento
				dall'inizio
				\item annullamento: comunicare a tutti i membri la decisione presa e ripetere il
				procedimento dall'inizio
			\end{itemize}
	\section{Svolgimento riunione}
		\subsection{Esterna}
			All'inizio di ogni riunione, verificata la presenza dei membri previsti, viene scelto un
			segretario che avrà il compito di annotare gli argomenti trattati e di redigere il verbale
			della riunione, che dovrà poi essere inviato ai restanti membri del gruppo.
			Tutti i partecipanti devono tenere un comportamento consono al miglior svolgimento
			dell'assemblea e al raggiungimento degli obbiettivi della stessa. Il segretario deve inoltre assicurarsi che venga seguito l'ordine del giorno in modo da affrontare ogni argomento previsto.
			\section{Verbale}
			\subsection{Riunione interna}
			Il verbale di Riunione interna è un documento informale che permette di tenere traccia
			degli argomenti discussi in ogni riunione.
			Il segretario, scelto a rotazione, ha il compito di redigere questo documento.
			Il verbale prodotto deve poi essere condiviso con tutti i membri del gruppo con Google
			Drive, in modo da rendere sempre disponibile e accessibile il contenuto dello stesso.
		\subsection{Riunione esterna}
			In caso di riunione esterna con il proponente e/o committente, il verbale è un documento
			ufficiale che può avere valore normativo e quindi deve essere redatto seguendo criteri
			specifici.
			Per agevolare la scrittura di tale documento è stato preparato un template \LaTeX\ che
			ne definisce la struttura e ne organizza i contenuti. Vi è quindi l'obbligo di seguire
			il sopraccitato schema per creare il verbale, di inviare il verbale prodotto come allegato al 
			proponente e/o committente in risposta all'email della riunione e di condividere il 
			verbale prodotto con tutti i membri del gruppo con Google Drive.
	
	\mychapter{4}{Documenti}
	\section{Template}
	Per rendere omogenea e semplice la stesura dei documenti e stato creato un template in \LaTeX\ che rispetta le regole stilistiche riportate in questo documento. Questo template e condiviso tramite github nella repository\ped{G}  documents\textbackslash templates 
	\section{Struttura del documento}
	\subsection{Prima pagina}
	La prima pagina di ogni documento riporta i seguenti campi:
	\begin{itemize}
		\item Nome del gruppo;
		\item Nome del progetto;
		\item Descrizione Progetto
		\item Titolo documento;
		\item Cognome e nome dei redattori del documento;
		\item Cognome e nome dei verificatori del documento;
		\item Cognome e nome dei responsabili del documento;
		\item Uso;
		\item Lista di distribuzione;
		\item Versione del documento;
		\item Data ultima modifica;
	\end{itemize}
	\subsection{Indici}
	In ogni documento viene usato un indice per le sezione e nel caso siano presenti anche per grafici e figure  
	\subsection{Diario delle modifiche}
	Subito dopo l'indice è presente una tabella che indica le modifiche apportate al documento ordinate in modo decrescente rispetto alla data.Per ogni riga viene indicato:
	\begin{itemize}
		\item Versione del documento;
		\item Data della modifica;
		\item Cognome Nome di chi ha apportqto la modifica;
		\item Descrizione della modifica;
	\end{itemize}
	\subsection{Struttura generale delle pagine}
	A piè pagina si trova il numero della pagina a partire dall'indice
	\section{Norme tipografiche}
	In questa sezione vengono descritte le regole di tipografia e ortografia comuni per tutti i documenti 
	\subsection{Punteggiatura}
	\begin{itemize}
		\item \textbf{Parentesi}: il testo racchiuso tra parentesi non deve iniziare o finire con spaziature inoltre alla fine non devono esserci presenti caratteri di punteggiatura;
		\item \textbf{Punteggiatura}:i caratteri di punteggiatura non devono mai essere preceduti da spaziatura;
		\item \textbf{Maiuscole}: l'iniziale maiuscola viene usata per il nome del team, del progetto, dei documenti, dei ruoli, delle varie fasi di lavoro e per le parole Proponente e Committente. \\Inotre viene usata negli elenchi puntati e nei casi indicati dalla lingua italiana.
	\end{itemize}
	
	\subsection{Stile di testo}
	\begin{itemize}
		\item \textbf{Corsivo}: usato nei seguenti casi:
		\begin{itemize}
			\item \textbf{Citazioni}: viene usato il corsivo quando una frase viene citata;
			\item \textbf{Abbreviazioni}: viene usato per evidenziare abbreviazioni;  
			\item \textbf{Nomi particolari}: i nomi di figure particolari come \textit{Progettista} o \textit{Analista};  
			\item \textbf{Documenti}: il corsivo verrà usato per i nomi dei vari documenti.
		\end{itemize}
		
		\item \textbf{Grassetto}: usato nei seguenti casi:
		\begin{itemize}
			\item \textbf{Elenchi puntati}: viene usato il grassetto per parole o frasi chiave all'interno di un elenco.
			
		\end{itemize}			
		viene inoltre usato per parole o particolari passaggi importanti;
		\item \textbf{Maiuscolo}: usato soltatnto per acronimi o eventuali macro \LaTeX\ nei documenti;
		\item \textbf{\LaTeX }: per ogni riferimento a \LaTeX\ bisogna utilizzare il comando \textbackslash LaTeX.
		
	\end{itemize}
	
	\subsection{Composizione del testo}
	\begin{itemize}
		\item \textbf{Elenchi puntati}: prima lettera minuscola (salvo casi precedenti) e devono terminare con punto e virgola, tranne l'ultimo elemento che terminera con il punto;
		\item \textbf{Note a piè pagina}: prima lettera maiuscola e devono terminare con il punto.
	\end{itemize}
	\subsection{Formati}
	\begin{itemize}
		\item \textbf{Numeri}: uso dello standard SI/ISO 31-10; 
		\item \textbf{Percorsi}: deve essere usato il comando \LaTeX\ \textbackslash url per indirizzi web  o mail;
		\item \textbf{Ore}: le ore devono seguire lo standard ISO 8601 quindi espresse come :\\
		\begin{center}
			HH:MM 
		\end{center}  
		dove HH indica l'ora espressa tramite 2 cifre (0-23) e MM indica i minuti espressi sempre con 2 cifre (0-59);
		\item \textbf{Date}: le date devono seguire lo standard ISO 8601:2004 quindi saranno espresse come:
		\begin{center}
			AAAA-MM-GG
		\end{center}  
		dove AAAA indica 4 cifre per l'anno, MM indica 2 cifre per il mese e GG indica 2 cifre per il giorno;
		
		\item \textbf{Riferimenti Vari}: Per i vari riferimenti si dovranno usare:\\  i seguenti comandi \LaTeX: (che garantiscono la corretta scrittura, con la prima lettera di ogni parola che non sia una preposizione maiuscola).
		\begin{itemize}
			\item \textbf{Ruoli}: per i ruoli si dovrà usare \textbackslash role\{Nome del ruolo\};
			\item \textbf{Documenti}: per i documenti si dovrà usare \textbackslash doc\{Nome del documento\};
			\item \textbf{Revisione}: per le revisioni si dovrà usare \textbackslash rev\{Nome Revisione\};
			\item \textbf{Fasi del progetto}: per le fasi del progetto si dovrà usare \textbackslash fasi\{Nome Fase\}.
		\end{itemize}
	\end{itemize}
	e sono state definite le seguenti macro \LaTeX:
	\begin{itemize}
		\item \textbf{Nome del gruppo}: per il nome del gruppo definito come "KaleidosCode" si dovrà usare \textbackslash GRUPPO;
		\item \textbf{Nome del proponente}: per riferirsi al proponente come “Zucchetti spa” o con “Proponente”. si dovrà usare \textbackslash PROPONENTE;	
		\item \textbf{Nome del committente}: per riferirsi al committente come “Prof. Vardanega Tullio” o con “Committente”. si dovrà usare \textbackslash COMMITTENTE;
		\item \textbf{Nome del progetto}: per riferirsi al proponente come “SweDesigner”. si dovrà usare \textbackslash PROGETTO.
	\end{itemize}
	Inoltre i nomi di file senza percorso completo si dovrà scrivere usando il formato monospace\ped G e per scrivere nomi dei componenti si dovrà usare il formato "Cognome Nome". 
	
	
	\subsection{Sigle}
	per rendere più accessibili tabelle e diagrammi si dovranno usare (dove 
	necessario) le seguenti sigle:
	\begin{itemize}
		\item \textbf{AdR}:Analisi dei Requisiti;
		\item \textbf{GL}:Glossario;
		\item \textbf{NdP}:Norme di Progetto;
		\item \textbf{PdP}:Piano di Progetto;
		\item \textbf{PdQ}:Piano di Qualifica;
		\item \textbf{SdF}:Studio di Fattibilità;
		\item \textbf{ST}:Specifica Tecnica;
		\item \textbf{RA}:Revisione di Accettazione;
		\item \textbf{RP}:Revisione di Progettazione;
		\item \textbf{RQ}:Revisione di Qualifica;
		\item \textbf{RR}:Revisione dei Requisiti.
	\end{itemize}
	\section{Componenti grafiche}
	\subsection{Tabelle}
	Ogni tabella deve essere accompagnata da una descrizione che specifichi anche l'indice ad essa associata per renderla tracciabile all'interno del documento.
	\subsection{Immagini}
	Le immagini dovranno essere convertite in pdf prima di essere incorporate nei documenti e ciascuna dovrà essere accompagnata da una descrizione e l'indice ad essa associata.  
	
	
	\section{Classificazione documenti}
	
	\subsection{Documenti formali}
	Un documento è ritenuto formale solo dopo essere stato approvato dal Responsabile di Progetto, dopo l'approvazione viene considerato pronto per la revisione da parte del Committente.
	\subsection{Documenti informali}
	Un documento è ritenuto finormale fino all' approvazione del Responsabile di Progetto, fino a quel momento viene considerato ad uso interno. 
	
	\section{Versionamento}
	Ogni documento dovrà essere accompagnato dal numero della versione attuale così formato:
	\begin{center}
		vX.Y.Z
	\end{center}
	dove:
	\begin{itemize}
		\item \textbf{X} indica il numero di uscite formali del documento e  aumenta a ogni approvazione da parte del responsabile;
		\item \textbf{Y}: usato per modifiche sostanziali e verifiche;
		\item \textbf{Z}: usato per indicare modifiche minori. 
	\end{itemize}
	Per riferirsi a una specifica versione del documento dovremo usare il seguente formato:\\
	\begin{center}
		\textit{Nome Documento vX.Y.Z}
	\end{center}
	mentre il nome da applicare ai file sarà:
	\begin{center}
		NomeDocumento\_vX.Y.Z.pdf
	\end{center}
	\subsection{Variazione indici}
	Le variazione degli indici avvengono da parte di:
	\begin{itemize}
		\item \textbf{X}: dovrà essere aggiornata dal responsabile dopo la sua approvazione ;
		\item \textbf{Y}: dovrà essere incrementato da chi esegue la modifica o la verifica;
		\item \textbf{Z}: dovrà essere incrementato da chi esegue la modifica.
	\end{itemize}
	queste modifiche verranno mostrate nel diario delle modifiche.
	
	\mychapter{5}{Analisi dei requisiti}
	
	\mychapter{6}{Codifica dei file e documentazione}
	
	\mychapter{7}{Glossario}
	
	\mychapter{8}{Protocollo per lo sviluppo dell'applicazione}
	
	\mychapter{9}{Ambiente di lavoro}
		
	\cleardoublepage
	\addcontentsline{toc}{chapter}{\listfigurename}
	\listoffigures
	
	\cleardoublepage
	\addcontentsline{toc}{chapter}{\listtablename}
	\listoftables
		
\end{document}
	